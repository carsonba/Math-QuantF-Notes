\documentclass[../main.tex]{subfiles}
\begin{document}



\section{Algebra}












\begin{definition}\label{def:algebraic number}
A number is called an \textit{algebraic number} if it satisfies a polynomial equation
\[
c_nx^n + c_{n-1}x^{n-1} + \dotsc + c_1x + c_0 = 0
\]
where the coefficients \(c_0, c_1, \dots, c_n\) are integers and \(c_n \neq 0\) and \(n \geq 1.\)
\end{definition}























\begin{theorem}[Rational Zeros Theorem]\label{thm:rational zeros theorem}
Suppose \(c_0, c_1, \dots , c_n \) are integers and \( r \in \mathbb{Q}\) satisfies the polynomial
\[
c_nx^n + c_{n-1}x^{n-1} + \dotsc + c_1x + c_0 = 0
\]
where \(n \geq 1, c_n \neq 0, \textnormal{ and } c_0 \neq 0.\) Let \(r = \frac{m}{d}, \textnormal{ where } m, d \in \mathbb{Z} \textnormal{ such that } \gcd(m,d) = 1 \textnormal{ and } d \neq 0.\) Then \(m\mid c_0 \textnormal{ and } d \mid c_n.\) 
\end{theorem}




\begin{proof}
Let \( x = r = m/d \) be a solution to the polynomial. Then,
\begin{gather*}
    c_nx^n + c_{n-1}x^{n-1} + \dotsc + c_1x + c_0 = 0, \\
    c_n\left(\frac{m^n}{d^n}\right) + c_{n-1}\left(\frac{m^{n-1}}{d^{n-1}}\right) + \dotsc + c_1\left(\frac{m}{d}\right) + c_0 = 0. \\
    c_n m^n + c_{n-1}m^{n-1}d + \cdots + c_1 m d^{n-1} + c_0 d^n = 0
\end{gather*}
Then rearranging, we see
\begin{gather*}
c_0 d^n = -m \left(c_n m^{n-1} + c_{n-1}m^{n-2} d + \cdots c_1  d^{n-1} \right)
\end{gather*}
Since \( gcd(m,d) = 1\), we know that \( gcd(m,d^n) = 1\), and thus \( m \) divides \( c_ 0\). Now rearranging again, we see
\begin{gather*}
c_n m^n = -d \left(c_{n-1} m^{n-1} + \cdots + c_1 m d^{n-2} + c_0 d^{n-1} \right)
\end{gather*}
Thus, \(d\) divides \(c_n\).
 
\end{proof}









\begin{remark}
    The result above states that given a polynomial with integer coefficients, a constant term, and a nonzero leading coefficient, if the polynomial is going to have rational roots, then the numerator of the root will divide the constant and the denominator will divide the leading coefficient. Note that often the leading coefficient is \(1\) so we typically only ensure the numerator divides the constant. Also note that we are not saying this rational is always a root, we are only saying that if a rational is a root, it has the form described above.
\end{remark}








\subsection{Divisibility in $\mathbb{Z}$}
We start by defining the integers. This ordered set will be our object of study. SAY MORE HERE

\begin{definition}[$\mathbb{Z}$]
The set of integers is any ordered set equipped with two operations $+, \cdot$ that satisfy the following axioms. $\forall a,b,c \in \mathbb{Z}:$
\begin{enumerate}
    \item If $a, b \in \mathbb{Z}$, then $a + b \in \mathbb{Z}$ \hfill \textit{[Closure for addition]}
    \item $a + (b + c) = (a + b) + c$ \hfill \textit{[Associative addition]}
    \item $a + b = b + a$ \hfill \textit{[Commutative addition]}
    \item $a + 0 = a = 0 + a$ \hfill \textit{[Additive identity]}
    \item For each $a \in \mathbb{Z}$, the equation $a + x = 0$ has a solution in $\mathbb{Z}$.
    \item If $a, b \in \mathbb{Z}$, then $ab \in \mathbb{Z}$ \hfill \textit{[Closure for multiplication]}
    \item $a(bc) = (ab)c$ \hfill \textit{[Associative multiplication]}
    \item $a(b + c) = ab + ac$ and \newline
          \hspace{0.5cm} $(a + b)c = ac + bc$ \hfill \textit{[Distributive laws]}
    \item $ab = ba$ \hfill \textit{[Commutative multiplication]}
    \item $a \cdot 1 = a = 1 \cdot a$ \hfill \textit{[Multiplicative identity]}
    \item If $ab = 0$, then $a = 0$ or $b = 0$.







    
\end{enumerate}

\end{definition}



\begin{remark}
    The below result is foundational to all of number theory and abstract algebra. It is the idea that given some number \(a\) to know how \(b\) \textit{fits into} \(a\) we will take as many copies or multiples of \(b\). We want to show existence and uniqueness. To show existence, we will show that such an \(r\) satisfying the hypothesis exists.\\
    So we will consider numbers of the form \(r = a-bq\). So we make a set of this form and show that it is nonempty. Then we will let the unique \(q, r\) correspond to the min of the set.
\end{remark}











\begin{theorem}[Division Algorithm]\label{thm:division_algorithim}
Let \(a,b \in \mathbb{Z}\) with \(b>0\). Then there exist unique \(q,r \in \mathbb{Z}\) such that \[
a = bq+r \textnormal{ and } 0 \leq r < b.
\]
\end{theorem}




\begin{proof}
Consider, 
\[
S = \{a - bx \mid \forall x \in \mathbb{Z}\}
\]
We start by showing \(S\) is nonempty.\\
Observe that \(|a| \in S \) since we can let \(x = 0\) which gives \(0 \leq a\), which tells us positive \(a\) is in \(S\).\\
Now let \(r = \min S\). We know \(r\) exists by the Well Ordering Axiom. Then let \(x = q\) correspond to \(r\). We will now show that \(r < b\).\\
By contradiction, suppose \(r > b\). Then this gives us that there is at least one factor of \(b\) in \(r\).
\[
a = bq + r = b(q+1) + r' \implies r' \in S \textnormal{ and } r' < r\]
which contradicts that \(r = \min S\), thus \(q\) and \(r\) exist.\\
Now we show uniqueness. Suppose there exists \(r' \textnormal{ and }q'\) such that \[
a = bq + r = bq' + r' \implies r' - r = b(q-q').
\]
Since we have that both \(r \textnormal{ and } r' \) are less than \(b\), this gives 
\[
\mid r' - r \mid < b \implies \mid b(q - q') \mid < b \implies \mid q - q' \mid < 1
\]
Then since the difference \(q - q'\) is an integer, we have that \(q = q' \implies r = r'\).
\end{proof}

\begin{definition}[Greatest Common Divisor]\label{def:gcd}
For any two nonzero integers \(a \textnormal{ and } b\), the \textit{greatest common divisor} \(gcd(a,b)\) is the unique positive integers \(d\) such that
\begin{enumerate}
    \item \(d \mid a \textnormal{ and } d \mid b\)
    \item If \( \exists c \in \mathbb{Z} \textnormal{ such that } c \mid a \) and \(c \mid b\), then \(c \leq d\).
\end{enumerate}
    
\end{definition}

\begin{remark}
    The greatest common divisor between any two integers will prove to be an important topic. When broken down, it is essentially a set of the shared factors of \(a\) and \(b\). Why would that be so? Because if \(d\) is the greatest magnitude greater than \(0\) that divides both \(a\) and \(b\) then every other divisor that is greater than 0 but be \textit{contained} in the magnitude of \(d\). This will be helpful as a sort of relation between the integers and their \textit{intersection with respect to divisibility}.
\end{remark}



















\begin{theorem}[Bezout's Identity]\label{thm:bezout_identity}
Let \(a\) and \(b\) be integers, not both 0, and let \(d = gcd(a,b)\). Then there exists integers \(u\) and \(v\) such that \[
gcd(a,b) = d = au + bv
\]  
\end{theorem}

\begin{remark}
\textit{Why would this make sense?} So recall that the \(gcd\) is the largest \textit{positive} divisor, then it would be plausible that the smallest positive integer linear combination of \(a\) and \(b\) is largest factor that is shared amongst \(a\) and \(b\). That is, through linear combinations, we can remove the multiples and factors of \(a\) and \(b\) that they dont have in their \textit{intersection}, then the magnitude that remains would be the \(gcd\). Also notice the usefulness of this result. This allows us to relate the divisibility structure of \(a\) and \(b\) to any combination that is made with them. \textit{Why does the \(gcd\) have to be the least positive element?} First consider if the smallest positive linear combination was greater than, say, \(a\). Since the \(gcd\) divides both \(a\) and \(b\),  the smallest linear combo must be less than both \(a \) and \(b\). If the smallest linear combo was smaller than the \(gcd\) then we would have that factors of \(a\) and \(b\) combine to something positive but less than the greatest factor they have in common. 
\end{remark}

\begin{proof}
Let \( S = \{au + bv \mid u,v \in \mathbb{Z}\}\). We will first show that \(S\) contains positive integers. Let \( u = a \textnormal{ and } v = b\), then we have \(a^2 + b^2 \in S\). Thus there exists positive integers in \(S\). Let \( t = \min S\), which we know exists because \(S \subset \mathbb{Z}\) so by well ordering axiom there must exist a least positive element. Define \(d = gcd(a,b) \). We want to show that \( t = d \). We will start by showing \( t \mid a\) and \(t \mid b\).
\[
\textnormal{By \ref{thm:division_algorithim},} \quad a = tq + r \implies r = a - tq \implies r = a - aqu - bqv \implies r = a(1-qu) + b(-qv)
\]
Thus \(r \in S\), but since, by the hypothesis of \ref{thm:division_algorithim} \quad \(r < t = \min S\). This implies that 

\end{proof}

\begin{remark}
So we hypothesized that the \(gcd\) was going to be a linear combination of \( a \) and \(b\)  because it is the greatest factor of them both, so in a way, they can both construct it. We then hypothesized in the proof that the \(gcd\) is the least positive multiple. So to show that \(t\) is the \(gcd\), we show that it divides them both and is the greatest such integer to do so. To show that \(t\) divides \( a \) and \(b\), we show, using \ref{thm:division_algorithim}  that the remainder must be in \(S\), but that would mean the remainder is less than \(t\) so that gives us what we are looking for. 
\end{remark}


\begin{proposition}\label{prp:gcd_divides_lin_combo}
Let \(a,b,x,y \in \mathbb{Z}\). Then 
\[
ax + by = c \iff gcd(a,b) \mid c.
\]
\end{proposition}

\begin{proof}
Suppose \(ax + by = c\) and let \(d = gcd(a,b)\). Then
\[
\exists k, l \in \mathbb{Z} \textnormal{ such that } c = dkx + dly \implies d \mid c. 
\]
Conversely, assume \(d = gcd(a,b) \mid c\). That is, \(\exists k \in \mathbb{Z}\) such that \(dk = c\). Then 
\[
c = dk = a(kx) + b(ky) \implies \exists u,v \in \mathbb{Z},\ c = au + bv.
\]
This concludes the proof.
\end{proof}




















\begin{proposition}\label{prp:coprime_divisor_of_product}
Let \(a,b,c \in \mathbb{Z}\). If \(a \mid bc \) and \(\gcd(a,b) = 1\), then \(a \mid c\).
    
\end{proposition}

\begin{proof}
Suppose \(a \mid bc \) and \(gcd(a,b) = 1\). Then \(\exists k \in \mathbb{Z}\) such that \(ak = bc\). Also by \ref{thm:bezout_identity},\[
\exists u,v \in \mathbb{Z} \textnormal{ such that } 1 = au + bv
\]
\[
\implies c = acu + bcv \implies c = ac + akv.
\]
Thus \(a \mid c\).
\end{proof}

\begin{remark}
This proposition is insightful to how the \(gcd\) will be used often. Notice we have that \(a\) divides a product but it shares no factors with \(b\), who is also in the product. Thus the only factors it must share with the product must be with \(c\). So we would expect to have that \(a\) divides \(c\).
\end{remark}





\begin{exercise}
Let $ a,b,c \in \mathbb{Z}$. Suppose $ \gcd(a,b) = 1$. If $a|c$ and $b|c$, then $ab|c.$
\end{exercise}





\begin{exercise}
Let $ a,b,c \in \mathbb{Z}$. Then $ \forall t \in \mathbb{Z}$ all of the following hold
\begin{enumerate}
    \item $\gcd(a,b) = \gcd(a,b + at)$
    \item $\gcd(ta,tb) = t\gcd(a,b) \quad \textnormal{ for } t > 0$
    \item $\gcd(a, \gcd(b,c)) = gcd(gcd(a,b), c)$
    \item $\gcd(a,c) = 1 \implies \gcd(ab, c) = \gcd(b,c) $
\end{enumerate}

\end{exercise}




\begin{exercise}
Let $ a,b,c \in \mathbb{Z}$. If $\gcd(a,c) = 1 $ and $\gcd(b,c)=1 \textnormal{, then }\gcd(ab,c) = 1 $
\end{exercise}



\begin{exercise}
A positive integer is divisible by $3$ $\iff$ the sum of its digits is divisible by $3$.
\end{exercise}

\begin{theorem}\label{thm:prime_dividing_product}
Let \(p \in \mathbb{Z}\) with \(p \neq 0,1, -1.\) Then \(p\) is prime if and only if \(p\) has the following property
\[
\textnormal{whenever } p \mid bc, \textnormal{ then } p \mid b \textnormal{ or } p \mid c
\]
\end{theorem}

\begin{remark}
This is obvious in comparison to \ref{prp:coprime_divisor_of_product} since a prime is coprime to every integer. Thus we will lean on that proof heavily.
\end{remark}

\begin{proof}
Suppose \(p\) is prime and consider \(p \mid bc\). Since \(p\) is prime, if \(p \mid b\) then the theorem is proved, if \(p \nmid b\) then since \(p\) is prime, \(gcd(p,b) = 1\). By \ref{prp:coprime_divisor_of_product} this gives us that \(p \mid b\) or \(p \mid c\). \\
Conversely, by the contrapositive, suppose \(p\) is not prime. Then if \(p \mid bc \textnormal{ then to have } p \mid b \textnormal{ or } p \mid c \textnormal{ we would need that } gcd(p,b) = 1, \ \forall b \in \mathbb{Z}.\) But this would mean that \(p\) is prime.

\end{proof}


\begin{theorem}[Fundamental Theoerem of Arithmetic]\label{thm:fund_thm_of_arith}
Every integer \(n \neq 0, 1, -1\) has a unique prime factorization.
\end{theorem}

\begin{proof}
First we will show existence of the factorization.\\
Let \( S = \{ n \in \mathbb{N} \mid n > 1 \textnormal{ and } \nexists \textnormal{ primes } p_1p_2 \cdots p_n \textnormal{ such that } p_1p_2 \cdots p_n = n \}\). Then assume, by contradiction, that \(S\) is nonempty. Then by the well ordering axiom, let \(n = \min S\). Since \(n\) is not prime, \(\exists a, b \in \mathbb{Z}\) such that \(ab = n\). Then this means \(a \mid n\) and \(b \mid n \). Since \(a,b \leq n\), we have that \(a\) and \(b\) have prime factorizations. Thus \(n\) has a prime factorization. This proves the existence of a prime factorization for all integers.\\
Now we will show that this factorization is unique.\\
By
\end{proof}





\begin{exercise}
If $n>1$ has no positive prime faster less than or equal to $\sqrt{n}$, then $s$ is prime.
\end{exercise}

\begin{exercise}
$a|b \iff a^n | b^n$
\end{exercise}




\subsection{Congruence and Congruence Classes}
\begin{remark}
The concepts below intend to study the structure that arithmetic and divisibility have among the integers. We do this by making our object of focus the remainder that an integer leaves after being divided. If some integer $a$ leaves behind the same remainder as some other integer $b$ when divided by $n$, then their difference $a-b$ is divisible by n. If we use their unique representation from \ref{thm:division_algorithim}, then
\[
a - b = nq_1 + r - nq_2 - r = n(q_1 - q_2)
\]
Why do we care about the divisibility structure? We will soon see that what we see as divisibility among numbers can actually be abstracted and shown to be an example of a more general concept. The concepts discussed later will show that the properties we find out about the integers actually are very similar properties that the more general elements share with each other.  
\end{remark}

\begin{definition}[Congruence $\pmod{n}$]
Let $a,b,n \in \mathbb{Z}$ with $n>0$. Then $a$ is congruent to $b$ modulo $n$ if $n \mid a-b$. This is denoted $ a \equiv b \quad \pmod{n}$

\end{definition}

\begin{theorem}[Congruence $\in$ Equivalence Relations]
Let $n$ be a positive integer, then $\forall a,b,c \in \mathbb{Z}$,
\begin{enumerate}
    \item $ a \equiv a \pmod{n}$
    \item If $ a \equiv b  \pmod{n}$, then $ b \equiv a \pmod{n}$
    \item If $ a \equiv b \pmod{n}$ and $ b \equiv c \pmod{n}$, then $ a \equiv c \pmod{n}$.
\end{enumerate}

\end{theorem}



\begin{proof}
The proof of $(1)$ and $(2)$ is straightforward after seeing the proof of $(3)$. If $ a \equiv b \mod{n}$ and $ b \equiv c \mod{n}$ then we can write
\begin{gather*}
    \exists k,l \in \mathbb{Z}: \quad a - b = nk \quad \textnormal{ and } \quad b - c = nl \\
    \implies \quad b = a - nk \quad \textnormal{ and } \quad b = c + nl \\
    \implies \quad a - c = n(k+l).
\end{gather*}
Thus $ a \equiv c \mod{n}$.

\end{proof}







\begin{proposition}[Modulo Arithmetic]\label{prp:mod_arithmetic}
If $ a \equiv b \pmod{n}$ and $ c \equiv d \pmod{n}$, then 
\begin{enumerate}
    \item $ a+c \equiv b+d \pmod{n}$
    \item $ ac \equiv bd \pmod{n}$
\end{enumerate}
\end{proposition}


\begin{proof}
$(1):$ Since $ a \equiv b $ and $ c \equiv d$ we have, by definition, $ a - b = nk$ and $ c - d = nl $. Adding these, we obtain $ (a+c) - (b+d) = n(k+l) \implies a + c \equiv b + d$. \\
$(2):$ So we want $ ac \equiv bd $, or equivalently, we want to find $ k \in \mathbb{Z}$ such that $ ac - bd = nk$. Then to use the hypothesis we do, 
\[
ac - bd = ac - bc + bc - bd = c(a-b) + b(c-d) = c(nk) + b(nl) = n(ck+bl).
\]
Thus, $ ac \equiv bd \mod{n}$.

\end{proof}








\begin{definition}[Congruence Class]
Let $a, n \in \mathbb{Z}$ be integers with $n > 0$. The \textit{congruence class} of $a$ modulo $n$ (denoted $[a]$) is the set of all integers that are congruent to $a$ modulo $n$, that is,
\[
    [a] = \{b \mid b \in \mathbb{Z} \quad \text{and} \quad b \equiv a \pmod{n} \}.
\]
Recall $b \equiv a \pmod{n}$ means that $b - a = kn$ for some integer $k$ or, equivalently, that $b = a + kn$. Thus
\[
[a] = \{b \mid b \equiv a \pmod{n} \} = \{b \mid b = a + kn \text{ with } k \in \mathbb{Z} \} = \{a + kn \mid k \in \mathbb{Z} \}
\]
\end{definition}


\begin{theorem}[Congruence Class Equality]\label{thm:congruence_class_equality}
$a \equiv c \pmod{n}$ if and only if $[a] = [c]$.

\end{theorem}

\begin{proof}
Suppose $ a \equiv c$, we want to show that $ [a] \subset [c]$ and $ [c] \subset [a]$, so also suppose that $ x \in [a]$. Then by definition of $ [a]$, $x \in [a] \implies x \equiv a$, then by transitivity, we have that $ x \equiv a$ and $ a \equiv c \implies x \equiv c \implies x \in [c].$ Suppose instead that $x \in [c]$. Then again by transitivity we obtain that $ x \in [a]$. \\
Suppose $ [a] = [c]$. Then by definition of $[a]$, $a \equiv a $ but since $ [a] = [c]$, we have that $ a \equiv a \implies a \in [c] \implies a \equiv c$.
\end{proof}







\begin{corollary}\label{cor:congruence_classes_disjoint}
Two congruence classes modulo $n$ are either disjoint or identical.
\end{corollary}

\begin{proof}
If $[a]$ and $[c]$ are disjoint, there is nothing to prove. Suppose that $[a] \cap [c]$ is nonempty. Then there is an integer $b$ with $b \in [a]$ and $b \in [c]$. Then, by the definition of congruence class, $b \equiv a \pmod{n}$ and $b \equiv c \pmod{n}$. Therefore, by symmetry and transitivity, $a \equiv c \pmod{n}$. Then by \ref{thm:congruence_class_equality} we have that, $[a] = [c]$. 
\end{proof}




\begin{exercise}
Let $n > 1$ be an integer and consider congruence modulo $n$.
\begin{enumerate}
    \item If $a$ is any integer and $r$ is the remainder when $a$ is divided by $n$, then $[a] = [r]$.
    \item There are exactly $n$ distinct congruence classes, namely, $[0], [1], [2], \dots, [n - 1]$.
\end{enumerate}
\end{exercise}



\begin{proof}
$(1):$ Suppose $a$ is an integer and $r$ is the remainder when $a$ is divided by $n$, then from \ref{thm:division_algorithim} we have, $ a = nk + r$ or $ a - r = nk \implies a \equiv r \implies [a] = [r]$.
Where the last implication used \ref{thm:congruence_class_equality}.\\
$(2):$ From $(1)$ we know that any given integer will be the same congruence class as its remainder $r$ where $ 0 \leq r < n$, thus there are $n-1$ such possible remainders. We also have from \ref{cor:congruence_classes_disjoint} that each class is disjoint, thus there are $n-1$ possible equivalence classes. 

\end{proof}

\begin{definition}
The set of all congruences classes modulo $n$ is denoted $\mathbb{Z}_n$.\\
Note that an element of $\mathbb{Z}_n$ is a class, the set of integers that it is congruent to, not a single integer. 
\end{definition}

\begin{exercise}
If $a, b$ are integers such that $a \equiv b \pmod{p}$ for every positive prime $p$, then $a = b$.
\end{exercise}




\begin{remark}
We will continue to study division in the integers at this abstracted level by using the concept that equivalence is defined by having the same remainder when divided by a number. The congruence class $\mathbb{Z}_n$ is a set consisting of other sets. These other sets are the sets of integers that are congruent modulo n, and the numbers that are congruent modulo n are the ones that have the same remainder when divided by n. Now we can define relations between classes more effectively.
\end{remark}




\begin{theorem}
If $[a] = [b]$ and $[c] = [d]$ in $\mathbb{Z}_n$, then
\[
[a + c] = [b + d] \quad \text{and} \quad [ac] = [bd].
\]
\end{theorem}

\begin{proof}
From \ref{thm:congruence_class_equality} we have that $ a\equiv b $ and $ c \equiv d$. Then from \ref{prp:mod_arithmetic} we have \[
a + c \equiv b+d \quad \textnormal{ and } \quad ac \equiv bd
\]
Then from \ref{thm:congruence_class_equality} again we have $ [a+c] = [b+d] $ and $ [ac] = [bd]$. \qed
\end{proof}




\begin{definition}[Operations in $\mathbb{Z}_n$]
We define addition $+$ and multiplication $ \cdot$ in $\mathbb{Z}_n$ by 
\[
 [a] \oplus [c] = [a + c] \quad \text{and} \quad [a] \odot [c] = [ac].
\]
\end{definition}











\begin{proposition}
For any classes $[a], [b], [c]$ in $\mathbb{Z}_n$,

\begin{enumerate}
    \item If $[a] \in \mathbb{Z}_n$ and $[b] \in \mathbb{Z}_n$, then $[a] \oplus [b] \in \mathbb{Z}_n$.
    \item $[a] \oplus ([b] \oplus [c]) = ([a] \oplus [b]) \oplus [c]$.
    \item $[a] \oplus [b] = [b] \oplus [a]$.
    \item $[a] \oplus [0] = [a] = [0] \oplus [a]$.
    \item For each $[a]$ in $\mathbb{Z}_n$, the equation $[a] \oplus x = [0]$ has a solution in $\mathbb{Z}_n$.
    \item If $[a] \in \mathbb{Z}_n$ and $[b] \in \mathbb{Z}_n$, then $[a] \odot [b] \in \mathbb{Z}_n$.
    \item $[a] \odot ([b] \odot [c]) = ([a] \odot [b]) \odot [c]$.
    \item $[a] \odot ([b] \oplus [c]) = [a] \odot [b] \oplus [a] \odot [c]$ and
          \newline \hspace{0.5cm} $([a] \oplus [b]) \odot [c] = [a] \odot [c] \oplus [b] \odot [c]$.
    \item $[a] \odot [b] = [b] \odot [a]$.
    \item $[a] \odot [1] = [a] = [1] \odot [a]$.
\end{enumerate}
\end{proposition}




\begin{remark}[Change of Notation]
From now on, to denote an element in $\mathbb{Z}_n$ we will just denote it by its integer form. That is, when we say we are \textit{in} $\mathbb{Z}_n$, then we will write $[a]_n$ as $a$. This is just for notational convenience, nothing has changed. 
\end{remark}



\begin{remark}
After some work with the integers modulo $n$, we start to notice a pattern, when the integers are modulo a prime number, the $\mathbb{Z}_n$ product of nonzero elements is always nonzero. So the distinction is that when $a\neq0$ the equation $ax = 1$ has a solution in $ \mathbb{Z}$ if and only if $ a = 1$ or $ a = -1$, but for the multiplication in $\mathbb{Z}_p $ where $p$ is a prime, the equation always has a solution. 
\end{remark}



\begin{theorem}
If $p > 1$ is an integer, then the following are equivalent:
\begin{enumerate}
    \item $p$ is prime.
    \item For any $a \neq 0$ in $\mathbb{Z}_p$, the equation $ax = 1$ has a solution in $\mathbb{Z}_p$.
    \item Whenever $bc = 0$ in $\mathbb{Z}_p$, then $b = 0$ or $c = 0$.
\end{enumerate}
\end{theorem}






\begin{corollary}
Let $a$ and $n$ be integers with $n > 1$. Then

The equation $[a]x = [1]$ has a solution in $\mathbb{Z}_n$ if and only if $\gcd(a, n) = 1$ in $\mathbb{Z}$.
\end{corollary}


\begin{definition}[Units]\label{def:units}
For any $ a \in \mathbb{Z}_n$, if $ \exists b \in \mathbb{Z}_n$ such that $ab = 1$, then $a$ is a \textit{unit}. In this case, we say $b$ is the \textit{inverse} of $a$.
\end{definition}

\begin{definition}[Zero Divisors]\label{def:zero_divisor}
Suppose $a \in \mathbb{Z}_n$ and $ a \neq 0$. If  $ \exists c \in \mathbb{Z}_n$ such that $c \neq 0$ and $ac = 0$.
\end{definition}




\begin{exercise} \label{ex:linear_congruences}
    Let \( n > 1 \) be an integer and let \( a, b \) be integers. Define \( d = \gcd(a, n) \). Consider the linear congruence
    \[
    [a]x = [b] \quad \text{in } \mathbb{Z}_n.
    \]
    
    \begin{enumerate}
        \item Show that the congruence has at least one solution if and only if \( d \mid b \). Conclude that no solution exists when \( d \nmid b \).
        
        \item Assume \( d \mid b \). Use Bézout’s identity to find integers \( u, v \) such that
        \[
        a u + n v = d.
        \]
        Show that
        \[
        x = \left[\frac{b}{d} u\right]
        \]
        is a solution in \( \mathbb{Z}_n \).
    
        \item Prove that every solution is of the form
        \[
        x = \left[\frac{b}{d} u + k \frac{n}{d}\right], \quad k \in \{0,1,\dots,d-1\}.
        \]
        Show that these \( d \) solutions are pairwise distinct.
    
        \item Conclude that if \( d \mid b \), there are exactly \( d \) distinct solutions, and otherwise, there are none. Explain how this fully classifies solutions to linear congruences.
    
        \item Solve the congruences:
        \[
        13x = 9 \quad \text{in } \mathbb{Z}_{24}, \quad \text{and} \quad 25x = 10 \quad \text{in } \mathbb{Z}_{65}.
        \]
        
        \item Show that if \( \gcd(a, n) = 1 \), then \( [a] \) is invertible in \( \mathbb{Z}_n \), ensuring a unique solution to \( [a]x = [b] \). Relate this to computing the inverse of \( [a] \) in \( \mathbb{Z}_n \).
    \end{enumerate}
    \end{exercise}
    






\subsection{Rings}
\par We now generalize the properties we have found consistent across the number-like systems we have studied. 


\begin{definition}[Ring]\label{def:ring}
A ring is a nonempty set R equipped with two operations \(+, \cdot\) that satisfy the following axioms. \(\forall a,b,c \in R\):
\begin{enumerate}
    \item If \(a \in R\) and \(b \in R\), then \(a + b \in R\). \hfill [Closure under Addition]
    \item \(a + (b+c) = (a+b)+c\) \hfill [Associativity of Addition]
    \item \(a + b = b + a\) \hfill [Commutativity of Addition]
    \item There exists an element \( 0_R \in R\) such that \(a + 0_R = a = 0_R + a, \ \forall a \in R\) \hfill [Additive identity]
    \item For each \(a \in R\), \(a + x = 0_R\) has a solution in \(R\), that is, \(x \in R\) \hfill [Additive Inverse]
    \item If \(a \in R\) and \(b \in R\), then \(ab \in R\) \hfill [Closure under Multiplication]
    \item \(a(bc) = (ab)c\) \hfill [Associativity of Multiplication]
    \item \( a(b+c) = ab + ac\) and \((a+b)c = ac + bc\) \hfill [Distributive Law] \\
The additional axioms below come from the definitions that are to follow. These definitions are the specific types of rings.
    \item $ab = ba \quad \forall a,b \in R$ \hfill [Commutative Ring] 
    \item $\exists 1_R \in R$ such that $a1_R = a = 1_Ra \quad \forall a \in R $. \hfill [Identity] 
    \item A commutative ring, with identity such that $ab=0 \implies a= 0 \textnormal{ or } b=0$. \hfill [Integral Domain] 
    \item A commutative ring, with identity such that $\forall a \neq 0 \in R$, $ax = 1$ has a solution in $R$. \hfill [Field] 
\end{enumerate}

\end{definition}


\begin{definition}[Commutative Ring]\label{def:commutative_ring}
A commutative ring is a ring \(R\) that satisfies the additional axiom: commutative multiplication
\[
ab = ba \quad \forall a,b \in R.
\]
\end{definition}

\begin{definition}[Multiplicative Identity] \label{def: multiplicative identity}
A ring with indentity is a ring \(R\) that contains an element \(1_R\) that satisfies the additional axiom: multiplicative identity
\[
a1_R = a = 1_R a \quad \forall a \in R.
\]
\end{definition}

\begin{definition}[Integral Domain]\label{def:integral domain}
An integral domain is a commutative ring \(R\) with identity \(1_R \neq 0_R\) that satisfies the additional axiom
\[
\textnormal{Whenever } a,b \in R \textnormal{ and } ab = 0_R, \textnormal{ then } a = 0_R \textnormal{ or } b = 0_R.
\]
    
\end{definition}


\begin{definition}[Field]\label{def:field}
A field is a commutative ring \(R\) with identity \(1_R \neq 0_R\) that satisfies the axiom 
\[
\textnormal{For each } a \neq 0_R \in R, \quad ax = 1_R \textnormal{ has a solution in \(R\)}
\]
\end{definition}


\begin{remark}
Note that these operations don't have to adhere to what we think of as addition and multiplication of two numbers....
\end{remark}



\begin{proposition}\label{prp:cartesian prod ring}
Let $R$ and $S$ be rings. Define addition and multiplication on the Cartesian product $R \times S$ by

\[
(r, s) + (r', s') = (r + r', s + s') \quad \text{and} \quad (r, s)(r', s') = (rr', ss').
\]

Then $R \times S$ is a ring. If $R$ and $S$ are both commutative, then so is $R \times S$. If both $R$ and $S$ have an identity, then so does $R \times S$.
\end{proposition}





\begin{theorem}[Subring] \label{thm:check for subring}
Suppose that $R$ is a ring and that $S$ is a subset of $R$ such that:
\begin{enumerate}
    \item $S$ is closed under addition (if $a, b \in S$, then $a + b \in S$);
    \item $S$ is closed under multiplication (if $a, b \in S$, then $ab \in S$);
    \item $0_R \in S$;
    \item If $a \in S$, then the solution of the equation $a + x = 0_R$ is in $S$.
\end{enumerate}
Then $S$ is a subring of $R$.
\end{theorem}

\begin{proof}
In order for $S$ to be a subring of $R$, we only need to check that the axioms for rings hold. 
Additonally, we need that the additive identity of $S$ is 
the same one that is in $R$. We need only check that 
axioms, from definition (\ref{def:ring}), 1,6,4, and 5 hold
since axioms 2,3,7, and 8 hold for all elements of $R$.
\end{proof}



\begin{theorem} \label{thm:uniqueness of additive inverse}
For any element $a$ in a ring $R$, the equation $a + x = 0_R$ has a unique solution.
\end{theorem}

\begin{proof}
From axiom 5 in definition (\ref{def:ring}), we know $a+x = 0_R$ has at least one solution, 
call it $u$. Then suppose $v$ is another solution. Then we have
\[
v = v + 0_R = v + (a+u) = (v+a) + u = 0_R + u = u.
\]
So $v=u$ and so the solution is always unique in any ring.
\end{proof}



\begin{theorem}\label{thm:subtraction}
If $a + b = a + c$ in a ring $R$, then $b = c$.
\end{theorem}

\begin{proof}
Using associativity from (\ref{def:ring}) we have
\[
a + c = a+b \implies (c+a)-a = (b+a)-a \implies c+(a-a) = b+(a-a)\implies b = c.
\]
\end{proof}



\begin{proposition} \label{prp:ring arithmetic with subtraction}
For any elements $a$ and $b$ of a ring $R$,
\begin{enumerate}
    \item $a \cdot 0_R = 0_R = 0_R \cdot a$. In particular, $0_R \cdot 0_R = 0_R$.
    \item $a(-b) = -ab$ \quad and \quad $(-a)b = -ab$.
    \item $-(-a) = a$.
    \item $-(a + b) = (-a) + (-b)$.
    \item $-(a - b) = -a + b$.
    \item $(-a)(-b) = ab$.
\end{enumerate}
If $R$ has an identity, then
\begin{enumerate}
    \setcounter{enumi}{6}
    \item $(-1_R)a = -a$.
\end{enumerate}
\end{proposition}

\begin{proof}
(1): Since $0 + 0 = 0$, using the distributive law, we have
\begin{gather*}
a \cdot 0 + a \cdot 0 = a(0 + 0) = a \cdot 0 = a \cdot 0 + 0 \\
\implies a \cdot 0 + a \cdot 0 = a \cdot 0 + 0 \implies a \cdot 0 = 0.
\end{gather*}
Note that the last implication uses \ref{thm:subtraction}. \\

(2): Since $-ab$ is the unique solution to $ab + x = 0$, any other solution
is equivalent to $-ab$ by \ref{thm:subtraction}. So we have\[
ab + a(-b) = a(b - b) = a\cdot 0 = 0 \implies -ab = a(-b).
\]\\
(3): Again from (\ref{thm:subtraction}) we know \(-(-a)\) is the unique solution
of \(-a + x = 0\), but \(a\) is also a solution, thus \(a = -(-a)\). \\
(4): Since \(-(a+b)\) is the unique solution of \((a+b) + x = 0\) and since addition
is commutative, we have \[
(a+b) + (-a) + (-b) = (a-a) + (b-b) = 0 + 0 = 0 \implies (-a) + (-b) = -(a+b).
\]\\
(5): By parts (4) and (3) above, we have \[
-(a-b) = (-a) + (-(-b)) = -a + b.
\]\\
(6): By parts (2) and (3) above, \[
(-a)(-b) = -(a(-b)) = -(-ab) = ab
\]\\
(7): By (2), we have \[
(-1)a = -(1a) = -a.
\]
\end{proof}



\begin{exercise}\label{ex:def of exponential}
Let $n,m \in \mathbb{N}$, if \(R\) is a ring with $a\in R$, then
\begin{gather*}
    a^n = aaa\cdots a \quad (\textnormal{n factors})\\
    a^na^m = a^{m+n} \textnormal{ and } (a^m)^n = a^{mn}
\end{gather*}


\end{exercise}




\begin{remark}
    Now with subtraction formally defined, we can revisit theorem \ref{thm:subring}
    and see if we can find a simpler method for checking subrings.
\end{remark}


\begin{proposition}[Subring]\label{thm:check for subring_p2}
Let $S$ be a nonempty subset of a ring $R$ such that:
\begin{enumerate}
    \item $S$ is closed under subtraction (if $a, b \in S$, then $a - b \in S$);
    \item $S$ is closed under multiplication (if $a, b \in S$, then $ab \in S$).
\end{enumerate}
Then $S$ is a subring of $R$.
\end{proposition}


\begin{proof}
We will show that this is equivalent to the hypotheses of theorem \ref{thm:check for subring}.
This means we only need to show that closure under subtraction implies (1) $S$ is closed 
under addition, (2) $0 \in S$, and (3) if $a \in S$ then $x\in S$, where $a+x = 0$. \\
(2): Since $S$ is nonempty and is closed under subtraction, we have that
 $c \in S$ exists so that $c-c = 0 \in S$. Thus $0 \in S$.\\
(3): Since $-a$ is the solution of $a+x = 0$, we just need that $-a \in S$.
Again, since $S$ is closed under subtraction, we have $0 - a = -a \in S$.\\
(1) By part (3) above, we have that $-b \in S$, and so from closure of subtraction
 \(a-b \in S \implies a - (-b) = a+b \in S\). Where the equality used (\ref{prp:ring arithmetic with subtraction}).

\end{proof}


\begin{definition}\label{def:units p2}
An element $a$ in a ring $R$ with identity is called a \textit{unit} if there exists $u \in R$ such that $au = 1_R = ua$. In this case, 
the element $u$ is called the (multiplicative) inverse of $a$ and is
denoted $a^{-1}$. Note that we already defined this in \ref{def:units}.
\end{definition}







\begin{definition}\label{def:zero divisor p2}
An element $a$ in a ring $R$ is a \textbf{zero divisor} provided that:
\begin{enumerate}
    \item $a \neq 0_R$.
    \item There exists a nonzero element $c$ in $R$ such that $ac = 0_R$ or $ca = 0_R$.
\end{enumerate}
Note that we already defined this in \ref{def:zero_divisor}.
\end{definition}




\begin{theorem}\label{thm:cancellation of Multiplication}
Cancellation is valid in any integral domain $R$: If $a \neq 0_R$ and $ab = ac$ in $R$, then $b = c$.
\end{theorem}

\begin{proof}
Since $ab=ac$ and since all rings are closed under subtraction (\ref{thm:subtraction}) we have \[
ab - ac = a(b-c) = 0
\]
since $S$ is an integral domain (\ref{def:integral domain}) we have $a=0$ or $b-c = 0$, but by hypothesis $a \neq 0$, thus $b=c$.
\end{proof}

\begin{theorem}\label{thm:fields are integral domains}
Every field $F$ is an integral domain.
\end{theorem}


\begin{proof}
Since both fields and integral domains are commutative rings with identity,
we only need to show that the existence of a solution $x \in R$ in $ax=1$ implies
that whenever $ab=0$ then $a=0$ or $b=0$. Suppose $b\neq 0$ and $ab=0$. By definition
(\ref{def:field}) we have $b^{-1} \in R$ such that $bb^{-1} = 1$. Then
\[
a = a1 = a(bb^{-1}) = (ab)b^{-1} = 0b^{-1} = 0
\].
\end{proof}












\begin{theorem}\label{thm:finite integral domain is field}
Every finite integral domain $R$ is a field.
\end{theorem}

\begin{proof}
Since $R$ is an integral domain (\ref{def:integral domain}) it has no zero divisors (\ref{def:zero divisor p2}).
Let $R' = R \ {0}$ and let $f: R' \to R'$ be the mapping $f(x) = ax$ for some fixed
$a \in R'$. Now if $f(x) = f(y)$ or $ax = ay$ then by cancellation for integral domains (\ref{thm:cancellation of Multiplication}) $x=y$,
thus $f$ is injective. But since $R (R')$ is finite, we have that $f$ is also surjective.
So fixing any $a \in R$, we have $\forall y \in R'$, $\exists x \in R'$ such that
$ax = y$. Letting $y = 1$ we see that $\forall a \in R'$, $\exists x \in R'$ such that $ax = 1$. 
\end{proof}



\begin{remark}
Consider the subset $\{0,2,4,6,8\}$ of $\mathbb{Z}_{10}$ along with
the set $\mathbb{Z}_5 = \{0,1,2,3,4,5\}$. Notice that the multiplication 
and addition amongst the subset of $\mathbb{Z}_10$ and amongst the elements in $\mathbb{Z}_5$
are analogous in that the only thing changing is the labels of the numbers. Meaning, 
for the elements of $\mathbb{Z}_5$, if we relabel 0 as 0, 1 as 6, 2 as 2, 3 as 8, and 4 as 4, we see
that the two sets are actually identical (after relabeling).\\
\indent The above is an example of having two structures and finding that for
however multiplication and addition are defined, every element along with the stucture those elements build (with operations)
can be paired off with elements of another structure. This is an isomorphism and is defined rigorously below.
\end{remark}



\begin{definition}[Isomorphism]
A ring $R$ is isomorphic to a ring $S$ (in symbols, $R \cong S$) if there is a function $f: R \to S$ such that all of the below hold:
\begin{enumerate}
    \item $f$ is injective;
    \item $f$ is surjective;
    \item $f(a + b) = f(a) + f(b)$ \quad and \quad $f(ab) = f(a) f(b)$ for all $a, b \in R$.
\end{enumerate}
In this case, the function $f$ is called an \textbf{isomorphism}.
\end{definition}




\begin{remark}
Now if we have that two rings are almost isomorphic but there does not 
exist a bijection amongst the elements, then we basically have only an isomorphism between
the structures only. This implies the operations, the things that build the structure, must satisfy the below.
\end{remark}

\begin{definition}[Homomorphism]
Let $R$ and $S$ be rings. A function $f: R \to S$ is said to be a \textbf{homomorphism} if
\[
f(a + b) = f(a) + f(b) \quad \text{and} \quad f(ab) = f(a) f(b) \quad \text{for all } a, b \in R.
\]
\end{definition}










\begin{theorem}\label{thm:homomorphism property}
Let $f: R \to S$ be a homomorphism of rings. Then
\begin{enumerate}
    \item $f(0_R) = 0_S$.
    \item $f(-a) = -f(a)$ for every $a \in R$.
    \item $f(a - b) = f(a) - f(b)$ for all $a, b \in R$.
\end{enumerate}
If $R$ is a ring with identity and $f$ is surjective, then
\begin{enumerate}
    \setcounter{enumi}{3}
    \item $S$ is a ring with identity $f(1_R)$.
    \item Whenever $u$ is a unit in $R$, then $f(u)$ is a unit in $S$ and $f(u)^{-1} = f(u^{-1})$.
\end{enumerate}
\end{theorem}

\begin{proof}
(1): Since $f$ is a homomorphism we have $f(0_R) + f(0_R) = f(0_R + 0_R) = f(0_R) = f(0_R) + 0_S$.
So this means, $f(0_R) = 0_S$.\\
(2): Let $a \in R$, then $f(a) + f(-a) = f(a-a) = f(0_R) = 0_S$ by (1).
Since $-f(a)$ is the solution to the equation $f(a) + x = 0_S$, and since 
we have that $f(-a)$ is also a solution. By \ref{thm:subtraction}, we have $-f(a) = f(-a)$.\\
(3): $ f(a-b) = f(a + (-b)) = f(a) + f(-b) = f(a) - f(b)$. Note that we used (2) on the third equality.\\
(4): Since $f$ is surjective, we have that $\forall s \in S$, $\exists r \in R$ such that
$s = f(r)$. Thus, \[
sf(1_R) = f(r)f(1_R) = f(r\cdot 1_R) = f(r) = s \implies f(1_R) = 1_S.
\]\\
(5): Using (4), we see that\[
1_S = f(1_R) = f(uu^{-1}) = f(u)f(u^{-1}) \implies f(u)f(u^{-1}) = 1_S
\]
So the multiplicative inverse of any $f(u)$ is $f(u^{-1})$ and since 
we denote the inverse of $f(u)$ as $f(u)^{-1}$, we see $f(u^{-1}) = f(u)^{-1}$.




\end{proof}











\begin{corollary}
If $f: R \to S$ is a homomorphism of rings, then the image of $f$ is a subring of $S$.
\end{corollary}

\begin{proof}
Denote the image of $f$ by $Im(f)$. $Im(f)$ is nonempty because 
$0_S = f(0_R) \in Im(f)$ by theorem \ref{thm:homomorphism property}.
Then by definition we have that if $f(a), f(b) \in Im(f)$ then 
$f(a)f(b) = f(ab) \in Im(f)$ and $f(a) - f(b) = f(a-b) \in Im(f)$, again by \ref{thm:homomorphism property}. Thus, $Im(f)$ is a subring of $S$ by \ref{thm:check for subring_p2}.

\end{proof}

\begin{remark}
Suppose there is some property amongst the elements of $R$ and their is 
an isomorphism between $R$ and $S$. Then we say the property is \textit{preserved}
under the isomorphism $f$ if that property is carried over, or, also seen in $S$.
For example, suppose $R$ is a commutative ring (\ref{def:commutative_ring}) and $f: R \to S$
is an isomorphism. Then $\forall a,b \in R$, we have $ab = ba \in R$. Therefore, in $S$ we have \[
f(a)f(b) = f(ab) = f(ba) = f(b)f(a).
\]
Which means $S$ is also a commutative ring. So we see here that the structure of commutative rings are preserved under isomorphisms.
\end{remark}















\end{document}