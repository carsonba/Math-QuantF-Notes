\documentclass[../main.tex]{subfiles}




\begin{document}






\section{Advanced Risk and Portfolio Management}


\subsection{Data Science}

\subsubsection{Probabilistic Framework}


















\subsubsection{Mean-Covariance Framework}
\indent In this framework, we model randomness by measuring only two characteristics of the random variable. We consider only the mean $E(\textbf{X})$ and the covarince $Cv(\textbf{X})$. The expectation gives us the location of our random variable in the multidimensional environment we model it in, and the covariance gives us the amount of dispersion in this random variable with each of the dimensions we define the space to be. Perhaps a better way of seeing this, is to notice that the first and second order terms of the taylor expansion of the characteristic function are fully characterized by the mean and covariance.\\
\indent We will class random variables based on their first two mooments, $\mu$ and $\sigma^2$. We will then consider affine (linear) transformations of the reference variable for a given class. For example, supposse we have a random variable $\mathbf{X}$ and we transform it into $\mathbf{Y} = \mathbf{a} + \mathbf{b}\mathbf{X}$ which amounts to a rotation, scaling, and translation of $\mathbf{X}$. Since the expectation is linear, this gives us the handy property seen below, the expectation will only act on $\mathbf{X}$,
\begin{equation}
    \underbrace{\left( \begin{array}
    [c]{c}\mathbb{E}\{Y_{1}\}\\
    \vdots\\
    \mathbb{E}\{Y_{\bar{k}}\}
    \end{array} \right) }_{\mathbb{E}\{\boldsymbol{Y}\}}=\underbrace{\left( \begin{array}
    [c]{c}a_{1}\\
    \vdots\\
    a_{\bar{k}}
    \end{array} \right) }_{\boldsymbol{a}}+\underbrace{\left( \begin{array}
    [c]{ccc}b_{1,1} & \cdots & b_{1,\bar{n}}\\
    \vdots & \ddots & \vdots\\
    b_{\bar{k},1} & \cdots & b_{\bar{k},\bar{n}}
    \end{array} \right) }_{\boldsymbol{b}}\underbrace{\left( \begin{array}
    [c]{c}\mathbb{E}\{X_{1}\}\\
    \vdots\\
    \mathbb{E}\{X_{\bar{n}}\}
    \end{array} \right) }_{\mathbb{E}\{\boldsymbol{X}\}}\text{.}
\end{equation}

So we will measure the returns of 
an asset at different intervals, then, in 
the mean covariance framework, we will characterize 
the \textit{random variables} based on their mean \ref{def:expected value} and covariance XXX REF Cov XXX.
For example, consider the linear returns $
    R_{t\rightarrow u}\equiv\frac{V_{u}}{V_{t}}-1\text{}$ of the $n=2$ assets
\begin{equation}
    \underbrace{\left( \begin{smallmatrix}
    X_{1}\\
    X_{2}
    \end{smallmatrix} \right) }_{\boldsymbol{X}}\equiv\left( \begin{smallmatrix}
    V_{1,t+1}/V_{1,t}-1\\
    V_{2,t+1}/V_{2,t}-1
    \end{smallmatrix} \right) \text{,}
    \end{equation}
where $V_{n,u}$ denotes the value of the $n$th asset at time $u$. \\
\indent We then start to measure and characterize the random variable $\textbf{X}$ of the returns, so we
find $\mathbb{E}(\textbf{X})$ and $\mathbb{C}v{\textbf(X)}$ where the expectation is a measure of the center
of the distrbution s

\[
    \mathbb{E}\{\boldsymbol{X}\}=\int_{\mathbb{R}^{\bar{n}}}\boldsymbol{x}dF_{\boldsymbol{X}}(\boldsymbol{x}) \quad \textnormal{or} \quad 
        \mathbb{E}\{\boldsymbol{X}\}\equiv\left( \begin{array}
        [c]{c}\mathbb{E}\{X_{1}\}\\
        \vdots\\
        \mathbb{E}\{X_{n}\}\\
        \vdots\\
        \mathbb{E}\{X_{\bar{n}}\}
        \end{array} \right) \text{,}
\]
the variance is given by 

\[
    \mathbb{V}\{X\}\equiv\mathbb{E}\{(X-\mathbb{E}\{X\})^{2}\}=\mathbb{E}\{X^{2}\}-\left( \mathbb{E}\{X\}\right) ^{2}\text{,}
    \quad \textnormal{ or } \quad 
    \mathbb{V}\{X\}=\int\nolimits_{-\infty}^{+\infty}(x-\mathbb{E}\{X\})^{2}dF_{X}(x)\text{.}
\]
and the covariance is given by

\[
    \mathbb{C}v\{X,Y\}\equiv\mathbb{E}\{(X-\mathbb{E}\{X\})(Y-\mathbb{E}\{Y\})\}=\mathbb{E}\{XY\}-\mathbb{E}\{X\}\mathbb{E}\{Y\}\text{,}
\]
\[
\mathbb{C}v\{X,Y\}=\int\nolimits_{\mathbb{R}^{2}}(x-\mathbb{E}\{X\})(y-\mathbb{E}\{Y\})dF_{X,Y}(x,y)\text{,}
\]
\[
    \mathbb{C}v\{\boldsymbol{X},\boldsymbol{Y}\}\equiv\left( \begin{array}
        [c]{cccc}\mathbb{C}v\{X_{1},Y_{1}\} & \mathbb{C}v\{X_{1},Y_{2}\} & \cdots & \mathbb{C}v\{X_{1},Y_{\bar{m}}\}\\
        \mathbb{C}v\{X_{2},Y_{1}\} & \mathbb{C}v\{X_{2},Y_{2}\} & \cdots & \mathbb{C}v\{X_{2},Y_{\bar{m}}\}\\
        \vdots & \vdots & \ddots & \vdots\\
        \mathbb{C}v\{X_{\bar{n}},Y_{1}\} & \mathbb{C}v\{X_{\bar{n}},Y_{2}\} & \cdots & \mathbb{C}v\{X_{\bar{n}},Y_{\bar{m}}\}
        \end{array} \right) \text{,}
\]
Notice in all of the expressions above we have one involving the cdf and one involving 
the expectation. Obviously the expecation is the mathematical definition where the integral is the application
of the definition. \\
Below we see the XXX REF z-score ZZZ which allows us to
indicate outliers in the data we work with.

\begin{equation}
    z_{X}(x)\equiv\dfrac{x-\mathbb{E}\{X\}}{\mathbb{S}\mathit{d}\left\{ X\right\} }\text{,}
    \end{equation}

The z-score is closely related to the signal to noise ratio 
\begin{equation}
    \mathbb{S}\mathit{n}\{Y\}=\frac{\mathbb{E}\{Y\}}{\mathbb{S}\mathit{d}\{Y\}}\text{.}
\end{equation}

Which is the usually the measure for a risk-return decison. When $X$ is the 
excess return over the risk free rate, the signal to noise ratio is the sharpe ratio
\begin{equation}
    \mathbb{S}\mathit{r}\{R\}\equiv\mathbb{S}\mathit{n}\{R-r\}\text{.}
    \end{equation}



Similarly, when $X$ is the excess return over the benchmark, 
the signal to noise ratio becomes the information ratio (obviously if the benchmark was the risk free rate IR = SR).

\begin{equation}
    \mathbb{I}\mathit{r}\{R\}\equiv\mathbb{S}\mathit{n}\{R-B\}\text{.}
    \end{equation}

\vspace{1cm}

We generalize the z-score to $n$ dimensional random variable $\mathbf{X}$ via 
the \textit{multivariate standard score} for an outcome $\mathbf{x}$

\begin{equation}
    z_{\boldsymbol{X}}\left( \boldsymbol{x}\right) \equiv(\mathbb{C}v\{\boldsymbol{X}\})^{-1/2}(\boldsymbol{x}-\mathbb{E}\{\boldsymbol{X}\})\text{,}
    \end{equation}

Where  $\mathbb{C}v\{\boldsymbol{X}\}^{-1/2}$  is a transpose square root $\mathbf{s}$, which is defined as the solution to the below
\begin{equation}
    \boldsymbol{s}\equiv\mathit{root}(\boldsymbol{s}^{2})\quad\Leftrightarrow \quad\boldsymbol{s}^{2}=\boldsymbol{s}\boldsymbol{s}^{\prime}\text{,}
    \end{equation}

Where the riccati root is the below, which is a result of the
XXXX REF spectral decomposition XXXX

\begin{equation}
    \boldsymbol{s}_{\mathit{Ricc}}=\mathit{root}_{\mathit{Ricc}}(\boldsymbol{s}^{2})\equiv\boldsymbol{e}\times\mathit{Diag}(\sqrt{\boldsymbol{\lambda}})\times\boldsymbol{e}^{\prime}\text{,}
    \end{equation}

Then the multivariate z score is given by 
\begin{equation}
    \Vert\boldsymbol{z}_{\boldsymbol{X}}(\boldsymbol{x})\Vert=\sqrt{(\boldsymbol{x}-\mathbb{E}\{\boldsymbol{X}\})^{\prime}(\mathbb{C}v\{\boldsymbol{X}\})^{-1}(\boldsymbol{x}-\mathbb{E}\{\boldsymbol{X}\})}\text{.}
    \end{equation}

This is the mahalanbis distance, given below, Where
the covariance matrix is scaling 
the multivariate distance property
of the mahalanobis.




























































\subsubsection{Linear Models}

\subsubsection{Machine Learning}



\subsubsection{Estimation}

\subsubsection{Inference}

\subsubsection{Sequential Decisions}

\subsection{Quantiative Finance}

\subsubsection{Financial Engineering}

\subsubsection{Risk Management}

\subsubsection{Portfolio Management}







\end{document}